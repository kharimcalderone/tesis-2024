\documentclass[11pt]{book}
\usepackage[letterpaper, left=2cm, right=2cm, top=0cm, bottom=2cm]{geometry}
\usepackage[spanish]{babel}  % Con este paquete se establece el idioma español
\usepackage[utf8]{inputenc} % Proporciona flexibilidad en cuento al uso de caracteres
\usepackage[T1]{fontenc}
\usepackage{comment}
\setcounter{secnumdepth}{3} %Profundidad de numeracion (ej 1, 1.1, 1.1.1, 1.1.1.1)

\setlength{\parindent}{5mm} % Modifica el grado de indenttación
\setlength{\parskip}{4mm} % Modifica la distancia entre parrafos

\usepackage{hyperref} %Crea hipervinculos


\begin{document}
\chapter{INTRODUCCIÓN}
\section{Introducción} \label{sec:S1}
La traducción automática (TA) representa una rama en constante evolución dentro del vasto panorama de la tecnología lingüística, dedicada a la creación y mejora de sistemas de software que posibilitan la traducción automatizada de textos entre diferentes lenguajes naturales. Este campo multidisciplinario se sumerge en la complejidad de los algoritmos y técnicas computacionales para desentrañar el significado y la estructura de un texto en el idioma de origen, allanando así el camino hacia la generación de traducciones coherentes y precisas en el idioma de destino. La tarea no solo abarca la simple sustitución de palabras, sino que también implica la comprensión contextual y semántica, así como la consideración de las sutilezas lingüísticas y culturales que enriquecen el proceso de traducción. Este desafío en constante evolución conlleva la exploración de diversas estrategias y enfoques innovadores para mejorar la calidad y la eficacia de la traducción automática, impulsando así el acceso global a la información y la comunicación intercultural.



\end{document}

